\documentclass{article}
\begin{document}

\subsection{Co-Inertia Analysis}

Co-Inertia Analysis (CoIA) emerged in ecology to facilitate analysis of
variation in species abundance as a function of environmental conditions
\citep{doledec1994co}. It can be viewed as a slight modification of CCA. Again,
we seek sets of orthonormal directions $\left(u_{k}\right)_{k = 1}^{K}$ and
$\left(v_{k}\right)_{k = 1}^{K}$ such that the associated projections $Xu_{k}$
and $Yv_{k}$ explain most of the covariation between the tables. Unlike CCA,
CoIA finds its first directions by maximizing the covariance -- not the
correlation -- between scores,
\begin{align*}
\maximize_{u \in \reals^{p_{1}}, v \in \reals^{p_{2}}}\medspace &u^{T}X^{T}Yv
\\ \text{ such that}\medspace &\|u\| = 1\\ &\|v\| = 1,
\end{align*}
with subsequent directions found by the same optimization, after adding the
constraint that they are orthogonal to the previously derived directions.

The only difference with the objective in equation \ref{eq:cancor_optim_emp} is
that norm constraint is imposed on $u$ and $v$ directly, rather than their
transformations $\Sigma_{XX}^{\frac{1}{2}}u$ and $\Sigma_{YY}^{\frac{1}{2}}v$.
It is in this sense that the CCA objective maximizes the correlation between
scores, while CoIA maximizes the covariance. The solutions
$\left(u_{k}\right)_{k = 1}^{K}$ and $\left(v_{k}\right)_{k = 1}^{K}$ can be
obtained as the first $K$ left and right eigenvectors from the SVD of $X^{T}Y$,
as opposed to the first $K$ generalized eigenvectors, as in CCA.

\subsubsection{Example}
\label{subsubsec:coia_example}

We apply CoIA to the same data as used in Section \ref{subsubsec:cca_example},
as CoIA also needs to estimate the covariance between tables, which is difficult
when the number of species is large. The loadings are displayed in Supplementary
Figure \ref{fig:coia_loadings}, and they are similar to those in Figure
\ref{fig:cca_loadings}. However, the associated scores are quite different than
those found using CCA. Compare Figure \ref{fig:coia_scores_android_fm}, which
shades samples by android fat mass, or Supplemental Figure
\ref{fig:coia_scores_rl_ratio}, which shades them by Bacteroides vs.
Ruminococcaceae differences, with Figures \ref{fig:cca_scores_android_fm} and
\ref{fig:cca_scores_rl_ratio} for CCA. The scores for CoIA are not so closely
aligned across tables, but they exhibit a clearer gradient across supplemental
variables. We find that the scores are not nearly as closely aligned as they are
for CCA (see Figure \ref{fig:cca_scores_android_fm}), but that they are more
strongly associated with variation in android fat mass, as in the concatenated
PCA result of Figure \ref{fig:scores_android_fm}. It is not clear whether this
phenomena -- the CoIA scores being more similar to those from PCA than CCA --
holds in general, or what it is about the change in inner products between CoIA
and CCA that is responsible for this difference.

\begin{figure}
  \centering
  \includegraphics[width=\textwidth]{figure/multitable/coia/scores_android_fm}
  \caption{The normalized scores from each table, displayed simultaneously, as
    obtained by CoIA. This is the analog of Figure
    \ref{fig:cca_scores_android_fm} from CCA, though the aspect ratio has been
    adjusted according to the relative amount of variance explained by the first
    two CoIA axes.
    \label{fig:coia_scores_android_fm} }
\end{figure}

\end{document}
