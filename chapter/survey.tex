\documentclass{article}
\usepackage{natbib}
\usepackage{amssymb, amsmath, amsfonts}
\usepackage{graphicx}
% ------------------------------------------------------------------------
% Packages
% ------------------------------------------------------------------------
\usepackage{amsmath}

% ------------------------------------------------------------------------
% Macros
% ------------------------------------------------------------------------
%~~~~~~~~~~~~~~~
% List shorthand
%~~~~~~~~~~~~~~~
\newcommand{\BIT}{\begin{itemize}}
\newcommand{\EIT}{\end{itemize}}
\newcommand{\BNUM}{\begin{enumerate}}
\newcommand{\ENUM}{\end{enumerate}}
%~~~~~~~~~~~~~~~
% Text with quads around it
%~~~~~~~~~~~~~~~
\newcommand{\qtext}[1]{\quad\text{#1}\quad}
%~~~~~~~~~~~~~~~
% Shorthand for math formatting
%~~~~~~~~~~~~~~~
\newcommand\mbb[1]{\mathbb{#1}}
\newcommand\mbf[1]{\mathbf{#1}}
\def\mc#1{\mathcal{#1}}
\def\mrm#1{\mathrm{#1}}
%~~~~~~~~~~~~~~~
% Common sets
%~~~~~~~~~~~~~~~
\def\reals{\mathbb{R}} % Real number symbol
\def\integers{\mathbb{Z}} % Integer symbol
\def\rationals{\mathbb{Q}} % Rational numbers
\def\naturals{\mathbb{N}} % Natural numbers
\def\complex{\mathbb{C}} % Complex numbers
\def\simplex{\mathcal{S}} % Simplex
%~~~~~~~~~~~~~~~
% Common functions
%~~~~~~~~~~~~~~~
\renewcommand{\exp}[1]{\operatorname{exp}\left(#1\right)} % Exponential
\def\indic#1{\mbb{I}\left({#1}\right)} % Indicator function
\providecommand{\maximize}{\mathop\mathrm{maximize}} % Defining math symbols
\providecommand{\maximize}{\mathop\mathrm{minimize}}
\providecommand{\argmax}{\mathop\mathrm{arg max}}
\providecommand{\argmin}{\mathop\mathrm{arg min}}
\providecommand{\arccos}{\mathop\mathrm{arccos}}
\providecommand{\asinh}{\mathop\mathrm{asinh}}
\providecommand{\dom}{\mathop\mathrm{dom}} % Domain
\providecommand{\range}{\mathop\mathrm{range}} % Range
\providecommand{\diag}{\mathop\mathrm{diag}}
\providecommand{\tr}{\mathop\mathrm{tr}}
\providecommand{\abs}{\mathop\mathrm{abs}}
\providecommand{\card}{\mathop\mathrm{card}}
\providecommand{\sign}{\mathop\mathrm{sign}}
\def\rank#1{\mathrm{rank}({#1})}
\def\supp#1{\mathrm{supp}({#1})}
%~~~~~~~~~~~~~~~
% Common probability symbols
%~~~~~~~~~~~~~~~
\def\E{\mathbb{E}} % Expectation symbol
\def\Earg#1{\E\left[{#1}\right]}
\def\Esubarg#1#2{\E_{#1}\left[{#2}\right]}
\def\P{\mathbb{P}} % Probability symbol
\def\Parg#1{\P\left({#1}\right)}
\def\Psubarg#1#2{\P_{#1}\left[{#2}\right]}
\def\Cov{\mrm{Cov}} % Covariance symbol
\def\Corr{\mrm{Corr}} % Covariance symbol
\def\Covarg#1{\Cov\left[{#1}\right]}
\def\Covsubarg#1#2{\Cov_{#1}\left[{#2}\right]}
\def\Corrsubarg#1#2{\Corr_{#1}\left[{#2}\right]}
\def\Var{\mrm{Var}}
\def\Vararg#1{\Var\left(#1\right)}
\def\Varsubarg#1#2{\Var_{#1}\left(#2\right)}
\newcommand{\family}{\mathcal{P}} % probability family
\newcommand{\eps}{\epsilon}
\def\absarg#1{\left|#1\right|}
\def\msarg#1{\left(#1\right)^{2}}
\def\logarg#1{\log\left(#1\right)}
%~~~~~~~~~~~~~~~
% Distributions
%~~~~~~~~~~~~~~~
\def\Gsn{\mathcal{N}}
\def\Ber{\textnormal{Ber}}
\def\Bin{\textnormal{Bin}}
\def\Unif{\textnormal{Unif}}
\def\Mult{\textnormal{Mult}}
\def\Cat{\textnormal{Cat}}
\def\Gam{\textnormal{Gam}}
\def\InvGam{\textnormal{InvGam}}
\def\NegMult{\textnormal{NegMult}}
\def\Dir{\textnormal{Dir}}
\def\Lap{\textnormal{Laplace}}
\def\Bet{\textnormal{Beta}}
\def\Poi{\textnormal{Poi}}
\def\HypGeo{\textnormal{HypGeo}}
\def\GEM{\textnormal{GEM}}
\def\BP{\textnormal{BP}}
\def\DP{\textnormal{DP}}
\def\BeP{\textnormal{BeP}}
%~~~~~~~~~~~~~~~
% Theorem-like environments
%~~~~~~~~~~~~~~~

%-----------------------
% Probability sets
%-----------------------
\newcommand{\X}{\mathcal{X}}
\newcommand{\Y}{\mathcal{Y}}
\newcommand{\D}{\mathcal{D}}
\newcommand{\Scal}{\mathcal{S}}
%-----------------------
% vector notation
%-----------------------
\newcommand{\bx}{\mathbf{x}}
\newcommand{\by}{\mathbf{y}}
\newcommand{\bt}{\mathbf{t}}
\newcommand{\xbar}{\overline{x}}
\newcommand{\Xbar}{\overline{X}}
\newcommand{\tolaw}{\xrightarrow{\mathcal{L}}}
\newcommand{\toprob}{\xrightarrow{\mathbb{P}}}
\newcommand{\laweq}{\overset{\mathcal{L}}{=}}
\newcommand{\F}{\mathcal{F}}


\title{Multitable Data Analysis for the Microbiome}
\author{Kris Sankaran}

\begin{document}
\maketitle

The simultaneous study of multiple measurement types is a commonly encoutered
problem in practical data analysis. It is especially common in microbiome
research, where several sources of data -- for example, 16s, metagenomic,
metabolomic, or transcriptomic data -- can be collected on the same physical
samples\citep{Franzosa2015, McHardy2013}. There has been a proliferation of
proposals for analyzing this kind of data, as is often the case when new (or
more cheaply available) data sources makes inquiry into certain scientific
questions all of a sudden possible \citep{Fukuyama2017, Rahnavard2017,
  Chaudhary2017, Chalise2017}.

However, stepping back from the rush of new methods for multiple table analysis
in the microbiome literature, it is worthwhile to recognize the broader
landscape of multiple table methods, as they have been relevant in problem
domains ranging from economics to robotics to computational biology. Of course,
there is no unique optimal algorithm to use across domains -- different
instances of the multiple table problem possess specific structure or variation
that are worth incorporating in methodology.

Our purpose here is not to develop new algorithms, but rather to (1) distill the
relevant themes across different analysis approaches and (2) provide concrete
workflows for approaching analysis, as a function of ultimate analysis goals and
characteristics (heterogeneity, dimensionality, sparsity, ...) of the data.
Towards the second goal, code for all analysis and figures are available.

First, though, why does anyone bother collecting multiple sources of data, and
why can't these sources simply be combined into a single, unified table for
subsequent analysis? One answer is simply that it's now cheaper and easier to
collect multitable data, but a more satisfactory explanation is that many
scientific problems can only be answered by collecting several complementary
measurement types. Indeed, the situation is analogous to using many types of
sensors to study a single system from many perspectives. Further, while in
certain supervised problems, it is enough to predict a single measurement of
interest, with other sources simply collected to provide better features, there
are often an additional relational components to the analysis: how do different
types of measurements covary with one another? Here, it is of interest to
provide a reduction / representation of the data that facilitates comparisons
across tables, rather than just comparing each table with a single response of
interest. This richer scientific question motivates the development of methods
distinct from those used to analyze a single measurement type at a time.

For more concrete motivation, we will use data from a study on the relationship
between various indicators of wellness and microbial community structure (todo:
give reference). In this study, 1969 individuals\footnote{Though sampling is
  still ongoing.} underwent clinical examinations, filled out wellness surveys
(covering topics such as exercise, sleep, diet, and mental health, for example),
and provided stool samples, used for 16s sequencing and metabolomic analysis.
To date, 16s sequencing data is available for 221 of these participants.
Evidently, various interesting relational questions can be investigated using
this data source.

For the purpose of illustration, we focus on one relatively focused question
that can be addressed using this data: How is the distribution of lean and fat
mass across the body related to patterns of microbial abundance? The measurement
types most relevant in this analysis are (1) DEXA scans and (2) 16s sequencing.
DEXA scans use relative X-ray absorption to gauge the among of lean and fat body
mass within the region of the body being scanned. We will have access to these
lean and fat body mass measurements at several body sites -- arms, legs, trunk,
etc. -- along with related body type variables, like height, age, and android
and gynoid fat measurements. In total, there are 33 of these variables. 16s
sequencing is a technology for gauging the abundance of differnet bacterial
species in the gut by counting the alignments of reads to the 16s gene, a
component of all bacterial genomes with enough variation to allow discimination
between different individual species. We have counts associated with 2565
species across 181 genuses, though the vast majority are present in low
abundances.

This question of the relationship between lean and fat mass distribution
(informally, body type) and the microbiome follows up findings that certain
taxonomic groups are over or underrepresented as a function of an individual's
BMI or obesity \citep{ley2006microbial, turnbaugh2009core, ley2005obesity,
  ley2010obesity}. Further, since the distribution of fat is often more related
to underlying biological mechanisms than overall body mass
\citep{matsuzawa2008role}, and since this distribution is mediated by specific
metabolic pathways, there is reason to suspect that a joint analysis of DEXA and
16s microbial abundance data might yield a more complete view of the
relationship between the microbiome and body type (and the associated disease
risk factors).

\section{Classical multivariate methods}

Methods from classical multivariate statistics are a mainstay of single-table
microbiome data analysis, so it is natural to revisit this literature before
surveying extensions to the multitable setting. Here we describe a few of the
classically studied multitable methods that fit nicely into the modern
microbiome data analysis toolbox. We first describe a naive approach based on
Principal Components Analysis (PCA) -- naive because it lifts a single-table
method to the multiple table setting without any special considerations --
before studying approaches that directly characterize covariation across several
tables: Canonical Correlation Analysis (CCA), Multiple Factor Analysis (MFA),
and Principal Component Analysis with Instrumental Variables (PCA-IV).

The earliest multitable method (CCA) was published in 1936, where the motivating
data analysis problem was to relate prices of groups of commodities
\citep{hotelling1936relations}. There are two notable aspects of data analysis in
this classical paradigm which no longer hold in modern statistics:
\begin{itemize}
  \item Even when many samples could be collected, there were typically only a
    few features for each sample, and it was straightforwards to study all of
    them simultaneously. In the last few decades, it has become possible to
    automatically collect a large number of features for each sample.
  \item Before electronic computers had been invented, it was important that all
    statistical quantities be easy to calculate, typically necessitating
    analytical formulas for parameter estimates. This is no longer as important
    a limitation in an environment with richer computational resources.
\end{itemize}

These changes have motivated the need for high-dimensional methods and
facilitated the adoption of iterative, more computationally-intensive
approaches, respectively, some of which are described later in this review.

Nonetheless, it is worth reviewing these original approaches, both to understand
the context for many modern techniques, as well as to have an easy starting
point for practical data analysis. Indeed, these more established methods tend
to be the most readily available through statistical computing packages and can
provide a benchmark with which to compare more elaborate, modern methods.

\subsection{PCA}
\label{sec:pca}

The simplest approach to dealing with multiple tables is to combine them into
one and apply a single-table method, for example, PCA. That is, write
\begin{align}
X = \left[X^{(1)} \vert \dots \vert X^{(L)}\right] \in \reals^{n \times p},
\end{align}
where $p = \sum_{l = 1}^{L}p_{l}$, and compute the SVD\footnote{An equivalent
  procedure is to eigenanalyze the empirical covariance matrix
  $\frac{1}{n}X^{T}X$.}, $X = UDV^{T}$. The $K$-principal component directions
are the first $K$ columns $v_{\cdot 1}, \dots, v_{\cdot K}$, while the
associated scores are $d_{1}u_{\cdot 1}, \dots, d_{K}u_{\cdot K}$.

While this does not account for the multitable structure of the data, it does
accomplish two goals,
\begin{itemize}
\item Through the principal component scores, it provides a visualization of the
  relationships between samples, based on all features.
\item Through the principal component directions, it gives a way of relating
  features within and across the multiple tables.
\end{itemize}

However, two drawbacks of this approach are worth noting,
\begin{itemize}
  \item It does not provide a summary of the relationship between the sets of
    variables defining the tables -- it can only relate pairs of
    variables. \label{bullet:pca_drawback_one}
  \item If some tables have many more variables than others, they can dominate
    the resulting ordination. \label{bullet:pca_drawback_two}
\end{itemize}

These limitations are addressed by CCA and MFA, discussed in Sections
\ref{sec:cca} and \ref{sec:mfa}, respectively.

Here we describe one geometric and one statistical motivation for PCA. The
geometric motivation is that, if each row $x_{i}$ of $X$ is viewed as a point in
$p$-dimensional space, then the principal component directions provide the best
$k$-dimensional approximation to the data, see Figure \ref{fig:pca-approx}.
Formally, recall that $VV^{T}x_{i}$ is the projection of $x_{i}$ onto the
subspace spanned by the columns of $V$. PCA identifies the orthogonal matrix $V
\in \reals^{p \times K}$ such that
\begin{align}
\sum_{i = 1}^{n}\|x_{i} - VV^{T} x_{i}\|_{2}^{2}
\end{align}
is minimized. The principal component scores are then the coordinates of the
projected points with respect to this subspace.

\begin{figure}
  \includegraphics[width=\textwidth]{figure/pca/proj_plot_1}
  \caption{A geometric motivation for PCA. The first principal component spans
    the one-dimensional subspace $V$ minimizing the squared error between points
    and their projections, $\sum_{i = 1}^{n}\|\left(I -
    VV^{T}\right)x_{i}\|_{2}^{2}$ projections. The black points represent the
    scaled values for two of the body composition variables -- weight and total
    fat mass -- while the purple points are the projections onto the first
    principal component when only accounting for these two variables. Contrast
    this with Figures \ref{fig:pca-approx-2} and \ref{fig:pca-approx-3}, where
    variables other than weight and fat mass are taken into account.}
  \label{fig:pca-approx}
\end{figure}

\begin{figure}
  \includegraphics[width=\textwidth]{figure/pca/var_plot}
  \caption{A statistical motivation for PCA. Each histogram is computed from the
    linear combinations $\left(c^{j}x_{i}\right)$, for $c^{j}$ equal to the
    first and second principal components, the vector $\frac{1}{n}\mathbf{1}$,
    and a normalized random multivariate Gaussian. At least among these four
    vectors, the scores obtained from the first PC have the largest variance. It
    is a theorem that its variance is in fact maximal among all linear
    combinations with orthogonal vectors.}
  \label{fig:pca-var}
\end{figure}

The second interpretation is that PCA finds a low-dimensional representation of
the $x_{i}$ such that the resulting points have maximal variance. Qualitatively,
this is a desirable property, because it means that the simpler representation
preserves most of the variation present in the original data, see Figure
\ref{fig:pca-var}. In this figure, histograms of the scores associated with four
linear combinations of the original body composition measurements are displayed
side by side, to emphasize the fact that the linear combination $x^{T}v_{1}$
associated with the first principal component $v_{1}$ has the largest variance.
The comparison combinations are the second principal component, the mean across
all columns, and a random linear combination $\frac{c^{T}}{\|c\|} x$ where $c \sim
\Gsn\left(0, I_{p}\right)$.

Formally, suppose that the $x_{i}\in\reals^{p}$ are drawn
independently from some distribution $\P$, so that the variance is
$\Covsubarg{\P}{x_{i}} = \Sigma$. Consider an arbitrary linear combination of
$x_{i}$'s $p$ coordinates: $z_{i} := c^{T}x_{i}$ for some $c \in \reals^{p}$.
The first PCA direction gives the $c$ such that the variance of this coordinate,
$\Varsubarg{\P}{z_{i}} = c^{T}\Sigma c$, is maximal. The second direction gives
the linear combination that maximizes variance, subject to being orthogonal to
the first, and so forth.

While our description of the method of concatenating multiple tables into a
single one has focused on PCA, note that other methods could be applied instead.
For example, it is possible to define a new distance between samples as a
mixture of distances based on several tables. This can be useful if there are
different types of data across the different tables: Jaccard, $\chi^{2}$, and
euclidean distsances can be applied to binary, count, and real valued tables.
The combined distance can then be input into any distance-based single-table
procedure, like multidimensional scaling or hierarchical clustering. The primary
downside of this approach is that the resulting distance only allows a
comparison between samples, but not across features.

PCA is a very widely used technique, and some standard references include
\citep{friedman2001elements, mardia1980multivariate, pages2014multiple}.
Nonetheless, it is not ideal in the multitable setting. Its limitations prompt
us to begin our survey of multitable methods in earnest.

\subsubsection{Example}
\label{subsec:pca_example}

\begin{figure}[ht]
  \centering
  \includegraphics[width=\textwidth]{figure/loadings}
  \caption{\label{fig:loadings} }
\end{figure}

\begin{figure}[ht]
  \centering
  \includegraphics[width=\textwidth]{figure/pca/scores_weight}
  \caption{\label{fig:scores_weight} }
\end{figure}

\begin{figure}[ht]
  \centering
  \includegraphics[width=\textwidth]{figure/pca/scores_rl_ratio}
  \caption{\label{fig:scores_rl_ratio} }
\end{figure}

\section{Appendix}

\subsection{Additional figures}

\begin{figure}
  \includegraphics[width=\textwidth]{figure/pca/proj_plot_2}
  \caption{A geometric motivation for PCA.}
  \label{fig:pca-approx-2}
\end{figure}

\begin{figure}
  \includegraphics[width=\textwidth]{figure/pca/proj_plot_3}
  \caption{A geometric motivation for PCA.}
  \label{fig:pca-approx-3}
\end{figure}

\bibliographystyle{plainnat}
\bibliography{refs.bib}

\end{document}
