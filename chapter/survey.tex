\documentclass{article}
\usepackage{natbib}
\usepackage{amssymb, amsmath, amsfonts}
\usepackage{graphicx}
% ------------------------------------------------------------------------
% Packages
% ------------------------------------------------------------------------
\usepackage{amsmath}

% ------------------------------------------------------------------------
% Macros
% ------------------------------------------------------------------------
%~~~~~~~~~~~~~~~
% List shorthand
%~~~~~~~~~~~~~~~
\newcommand{\BIT}{\begin{itemize}}
\newcommand{\EIT}{\end{itemize}}
\newcommand{\BNUM}{\begin{enumerate}}
\newcommand{\ENUM}{\end{enumerate}}
%~~~~~~~~~~~~~~~
% Text with quads around it
%~~~~~~~~~~~~~~~
\newcommand{\qtext}[1]{\quad\text{#1}\quad}
%~~~~~~~~~~~~~~~
% Shorthand for math formatting
%~~~~~~~~~~~~~~~
\newcommand\mbb[1]{\mathbb{#1}}
\newcommand\mbf[1]{\mathbf{#1}}
\def\mc#1{\mathcal{#1}}
\def\mrm#1{\mathrm{#1}}
%~~~~~~~~~~~~~~~
% Common sets
%~~~~~~~~~~~~~~~
\def\reals{\mathbb{R}} % Real number symbol
\def\integers{\mathbb{Z}} % Integer symbol
\def\rationals{\mathbb{Q}} % Rational numbers
\def\naturals{\mathbb{N}} % Natural numbers
\def\complex{\mathbb{C}} % Complex numbers
\def\simplex{\mathcal{S}} % Simplex
%~~~~~~~~~~~~~~~
% Common functions
%~~~~~~~~~~~~~~~
\renewcommand{\exp}[1]{\operatorname{exp}\left(#1\right)} % Exponential
\def\indic#1{\mbb{I}\left({#1}\right)} % Indicator function
\providecommand{\maximize}{\mathop\mathrm{maximize}} % Defining math symbols
\providecommand{\maximize}{\mathop\mathrm{minimize}}
\providecommand{\argmax}{\mathop\mathrm{arg max}}
\providecommand{\argmin}{\mathop\mathrm{arg min}}
\providecommand{\arccos}{\mathop\mathrm{arccos}}
\providecommand{\asinh}{\mathop\mathrm{asinh}}
\providecommand{\dom}{\mathop\mathrm{dom}} % Domain
\providecommand{\range}{\mathop\mathrm{range}} % Range
\providecommand{\diag}{\mathop\mathrm{diag}}
\providecommand{\tr}{\mathop\mathrm{tr}}
\providecommand{\abs}{\mathop\mathrm{abs}}
\providecommand{\card}{\mathop\mathrm{card}}
\providecommand{\sign}{\mathop\mathrm{sign}}
\def\rank#1{\mathrm{rank}({#1})}
\def\supp#1{\mathrm{supp}({#1})}
%~~~~~~~~~~~~~~~
% Common probability symbols
%~~~~~~~~~~~~~~~
\def\E{\mathbb{E}} % Expectation symbol
\def\Earg#1{\E\left[{#1}\right]}
\def\Esubarg#1#2{\E_{#1}\left[{#2}\right]}
\def\P{\mathbb{P}} % Probability symbol
\def\Parg#1{\P\left({#1}\right)}
\def\Psubarg#1#2{\P_{#1}\left[{#2}\right]}
\def\Cov{\mrm{Cov}} % Covariance symbol
\def\Corr{\mrm{Corr}} % Covariance symbol
\def\Covarg#1{\Cov\left[{#1}\right]}
\def\Covsubarg#1#2{\Cov_{#1}\left[{#2}\right]}
\def\Corrsubarg#1#2{\Corr_{#1}\left[{#2}\right]}
\def\Var{\mrm{Var}}
\def\Vararg#1{\Var\left(#1\right)}
\def\Varsubarg#1#2{\Var_{#1}\left(#2\right)}
\newcommand{\family}{\mathcal{P}} % probability family
\newcommand{\eps}{\epsilon}
\def\absarg#1{\left|#1\right|}
\def\msarg#1{\left(#1\right)^{2}}
\def\logarg#1{\log\left(#1\right)}
%~~~~~~~~~~~~~~~
% Distributions
%~~~~~~~~~~~~~~~
\def\Gsn{\mathcal{N}}
\def\Ber{\textnormal{Ber}}
\def\Bin{\textnormal{Bin}}
\def\Unif{\textnormal{Unif}}
\def\Mult{\textnormal{Mult}}
\def\Cat{\textnormal{Cat}}
\def\Gam{\textnormal{Gam}}
\def\InvGam{\textnormal{InvGam}}
\def\NegMult{\textnormal{NegMult}}
\def\Dir{\textnormal{Dir}}
\def\Lap{\textnormal{Laplace}}
\def\Bet{\textnormal{Beta}}
\def\Poi{\textnormal{Poi}}
\def\HypGeo{\textnormal{HypGeo}}
\def\GEM{\textnormal{GEM}}
\def\BP{\textnormal{BP}}
\def\DP{\textnormal{DP}}
\def\BeP{\textnormal{BeP}}
%~~~~~~~~~~~~~~~
% Theorem-like environments
%~~~~~~~~~~~~~~~

%-----------------------
% Probability sets
%-----------------------
\newcommand{\X}{\mathcal{X}}
\newcommand{\Y}{\mathcal{Y}}
\newcommand{\D}{\mathcal{D}}
\newcommand{\Scal}{\mathcal{S}}
%-----------------------
% vector notation
%-----------------------
\newcommand{\bx}{\mathbf{x}}
\newcommand{\by}{\mathbf{y}}
\newcommand{\bt}{\mathbf{t}}
\newcommand{\xbar}{\overline{x}}
\newcommand{\Xbar}{\overline{X}}
\newcommand{\tolaw}{\xrightarrow{\mathcal{L}}}
\newcommand{\toprob}{\xrightarrow{\mathbb{P}}}
\newcommand{\laweq}{\overset{\mathcal{L}}{=}}
\newcommand{\F}{\mathcal{F}}


\title{Multitable Data Analysis for the Microbiome}
\author{Kris Sankaran}

\begin{document}
\maketitle

\section{Classical multivariate methods}

Methods from classical multivariate statistics are a mainstay of single-table
microbiome data analysis, so it is natural to revisit this literature before
surveying extensions to the multitable setting. Here we describe a few of the
classically studied multitable methods that fit nicely into the modern
microbiome data analysis toolbox. We first describe a naive approach based on
Principal Components Analysis (PCA) -- naive because it lifts a single-table
method to the multiple table setting without any special considerations --
before studying approaches that directly characterize covariation across several
tables: Canonical Correlation Analysis (CCA), Multiple Factor Analysis (MFA),
and Principal Component Analysis with Instrumental Variables (PCA-IV).

The earliest multitable method (CCA) was published in 1936, where the motivating
data analysis problem was to relate prices of groups of commodities
\cite{hotelling1936relations}. There are two notable aspects of data analysis in
this classical paradigm which no longer hold in modern statistics:
\begin{itemize}
  \item Even when many samples could be collected, there were typically only a
    few features for each sample, and it was straightforwards to study all of
    them simultaneously. In the last few decades, it has become possible to
    automatically collect a large number of features for each sample.
  \item Before electronic computers had been invented, it was important that all
    statistical quantities be easy to calculate, typically necessitating
    analytical formulas for parameter estimates. This is no longer as important
    a limitation in an environment with richer computational resources.
\end{itemize}

These changes have motivated the need for high-dimensional methods and
facilitated the adoption of iterative, more computationally-intensive
approaches, respectively, some of which are described later in this review.

Nonetheless, it is worth reviewing these original approaches, both to understand
the context for many modern techniques, as well as to have an easy starting
point for practical data analysis. Indeed, these more established methods tend
to be the most readily available through statistical computing packages and can
provide a benchmark with which to compare more elaborate, modern methods.

\subsection{PCA}
\label{sec:pca}

The simplest approach to dealing with multiple tables is to combine them into
one and apply a single-table method, for example, PCA. That is, write
\begin{align}
X = \left[X^{(1)} \vert \dots \vert X^{(L)}\right] \in \reals^{n \times p},
\end{align}
where $p = \sum_{l = 1}^{L}p_{l}$, and compute the SVD\footnote{An equivalent
  procedure is to eigenanalyze the empirical covariance matrix
  $\frac{1}{n}X^{T}X$.}, $X = UDV^{T}$. The $K$-principal component directions
are the first $K$ columns $v_{\cdot 1}, \dots, v_{\cdot K}$, while the
associated scores are $d_{1}u_{\cdot 1}, \dots, d_{K}u_{\cdot K}$.

While this does not account for the multitable structure of the data, it does
accomplish two goals,
\begin{itemize}
\item Through the principal component scores, it provides a visualization of the
  relationships between samples, based on all features.
\item Through the principal component directions, it gives a way of relating
  features within and across the multiple tables.
\end{itemize}

However, two drawbacks of this approach are worth noting,
\begin{itemize}
  \item It does not provide a summary of the relationship between the sets of
    variables defining the tables -- it can only relate pairs of
    variables. \label{bullet:pca_drawback_one}
  \item If some tables have many more variables than others, they can dominate
    the resulting ordination. \label{bullet:pca_drawback_two}
\end{itemize}

These limitations are addressed by CCA and MFA, discussed in Sections
\ref{sec:cca} and \ref{sec:mfa}, respectively.

Here we describe one geometric and one statistical motivation for PCA. The
geometric motivation is that, if each row $x_{i}$ of $X$ is viewed as a point in
$p$-dimensional space, then the principal component directions provide the best
$k$-dimensional approximation to the data, see Figure \ref{fig:pca-approx}.
Formally, recall that $VV^{T}x_{i}$ is the projection of $x_{i}$ onto the
subspace spanned by the columns of $V$. PCA identifies the orthogonal matrix $V
\in \reals^{p \times K}$ such that
\begin{align}
\sum_{i = 1}^{n}\|x_{i} - VV^{T} x_{i}\|_{2}^{2}
\end{align}
is minimized. The principal component scores are then the coordinates of the
projected points with respect to this subspace.

\begin{figure}
  \caption{A geometric motivation for PCA.}
  \label{fig:pca-approx}
\end{figure}

\begin{figure}
  \caption{A statistical motivation for PCA.}
  \label{fig:pca-var}
\end{figure}

The second interpretation is that PCA finds a low-dimensional representation of
the $x_{i}$ such that the resulting points have maximal variance. Qualitatively,
this is a desirable property, because it means that the simpler representation
preserves most of the variation present in the original data, see Figure
\ref{fig:pca-var}. Formally, suppose that the $x_{i}\in\reals^{p}$ are drawn
independently from some distribution $\P$, so that the variance is
$\Covsubarg{\P}{x_{i}} = \Sigma$. Consider an arbitrary linear combination of
$x_{i}$'s $p$ coordinates: $z_{i} := c^{T}x_{i}$ for some $c \in \reals^{p}$.
The first PCA direction gives the $c$ such that the variance of this coordinate,
$\Varsubarg{\P}{z_{i}} = c^{T}\Sigma c$, is maximal. The second direction gives
the linear combination that maximizes variance, subject to being orthogonal to
the first, and so forth.

While our description of the method of concatenating multiple tables into a
single one has focused on PCA, note that other methods could be applied instead.
For example, it is possible to define a new distance between samples as a
mixture of distances based on several tables. This can be useful if there are
different types of data across the different tables: Jaccard, $\chi^{2}$, and
euclidean distsances can be applied to binary, count, and real valued tables.
The combined distance can then be input into any distance-based single-table
procedure, like multidimensional scaling or hierarchical clustering. The primary
downside of this approach is that the resulting distance only allows a
comparison between samples, but not across features.

PCA is a very widely used technique, and some standard references include
\cite{friedman2001elements, mardia1980multivariate, pages2014multiple}.
Nonetheless, it is not ideal in the multitable setting. Its limitations prompt
us to begin our survey of multitable methods in earnest.

\bibliographystyle{plainnat}
\bibliography{refs.bib}

\end{document}
